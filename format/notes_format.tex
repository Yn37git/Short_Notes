\documentclass[11pt]{extarticle}
\usepackage{fullpage}
\usepackage[ampersand]{easylist}
\ListProperties(Hide=10, Style*=$\bullet\;\,$, Style2*=$\;\,${\tiny$\blacksquare$}$\;\,$,Space*=1mm,Space2*=0.1mm)

\usepackage{hyperref}
\hypersetup{colorlinks=true,
	linkcolor=black,
	filecolor=black,      
	urlcolor=black}

\usepackage{amsmath,amssymb,amsthm,mathtools,mathrsfs}
\newtheorem{thm}{Theorem}[]
\usepackage{bibref}

\usepackage[T1]{fontenc}
\usepackage[sc,osf]{mathpazo}
\usepackage{eulervm}
\usepackage[bold=.05]{xfakebold}

\usepackage{tikz}
\usetikzlibrary{calc}
\usetikzlibrary{shapes}
\usepackage{pgfplots}
\pgfplotsset{trig format plots=rad}
\usetikzlibrary {3d}
\pgfplotsset{compat=1.18}

\usepackage[most]{tcolorbox}
\tcbuselibrary{skins}
\usepackage[explicit]{titlesec}
\newtcolorbox{secbox}[1][]{enhanced,attach boxed title to top center,drop fuzzy shadow,breakable,colbacktitle=gray,colback=black,colframe=black,
	coltext=white,size=title,title={#1}}
\titleformat{\section}[runin]{\bfseries\LARGE}{}{0pt}{\hfill
	%\begin{secbox}[\thesection]
	%	\centering #1
	%\end{secbox}
	%}
\tcbsidebyside[sidebyside adapt=left,segmentation style=solid,enhanced,size=small]
{%
	\thesection 
}
{%
	#1
}
}
\titleformat{\subsection}[runin]{\bfseries\large}{}{0pt}
{\hfill
%	\begin{secbox}[\thesubsection]
	%		\centering #1
	%	\end{secbox}
\tcbsidebyside[sidebyside adapt=left,segmentation style=solid,enhanced,size=small]
{%
	\thesubsection 
}
{%
	#1
}
}
\titleformat{\subsubsection}[runin]{\bfseries}{}{0pt}
{\hfill
%		\begin{secbox}[\thesubsubsection]
	%			\centering #1
	%		\end{secbox}
\tcbsidebyside[sidebyside adapt=left,segmentation style=solid,enhanced,size=small]
{%
	\thesubsubsection 
}
{%
	#1
}
}

\usepackage{multicol}
\setlength{\columnsep}{5mm}
\setlength\columnseprule{.1pt}

\newcommand{\ra}{\rightarrow}
\newcommand{\R}{\mathbb{R}}
\newcommand{\C}{\mathbb{C}}
\newcommand{\Na}{\mathbb{N}}
\newcommand{\Z}{\mathbb{Z}}
\newcommand{\Q}{\mathbb{Q}}
\newcommand{\w}[1]{\text{#1}}
\newcommand{\ck}{.\,.\,}
\newcommand{\sm}[2]{\displaystyle\sum_{#1}^{#2}}
\newcommand{\Uint}[2]{\overline{\int\!}_{#1}^{\;#2}}
\newcommand{\Lint}[2]{\underline{\int\!}_{\;#1}^{\;#2}}
\newcommand{\pfrac}[2]{\frac{\partial#1}{\partial#2}}
\newcommand{\ckfil}{$.\dotfill.$}
\newcommand{\tm}{\times}
\newcommand{\snote}[1]{{\footnotesize(#1)}}
\newcommand{\st}{\,{}_{s}|_t\,}
\newcommand{\gen}[1]{\langle #1 \rangle}
\newcommand{\tbx}[2][]{
\begin{tcolorbox}[enhanced,breakable,size=small,colback=black!2!white,title={#1},arc is angular, arc=1.5mm,drop fuzzy shadow]
	#2
\end{tcolorbox}
}
\newcommand{\y}{$\blacksquare\;$}

\author{Yashas.N}
\title{Title}
\date{}
\begin{document}
    \maketitle
    \boldmath
\begin{multicols}{2}
    \section{Section1}
    \tbx[Point1]{for $\{1,2\ck n \}=A_n\subset \Z^+  $ \\
    \y $ \sm{a\in A_n}{} a=\frac{ n(n+1) }{2} .  $\\
    \y $ \sm{a\in A_n}{} a^2 =\frac{n(n+1)(2n+1)}{6}.$  \\
    \y  $ \sm{a\in A_n}{} a^3 =\left(\frac{n(n+1)}{2}\right)^2.$   }
    
    \subsection{subsection1}
\tbx[Theorem]{ 	\begin{align*}\small
		&\sm{a\in A_n}{} a^k =\\
		&  \frac{ (n+1)^{k+1}-1-\sm{i=0}{k-1}\left({k+1\choose i}\sm{a\in A_n}{}a^{i}\right)}{{k+1}}.
		\end{align*}      } 
\begin{proof}

consider the following\\
\begin{align}
	2^{k+1}&= (1+1)^{k+1}=\sm{i=1}{k+1}{n+1\choose i}.\label{1}\\
	3^{k+1}&= (2+1)^{k+1}=\sm{i=1}{k+1}{k+1\choose i}2^i, \nonumber\\
	&=2^{k+1}+\sm{i=0}{k}{k+1\choose i}2^i.  \label{2}
\end{align}
Substituting \eqref{1} in \eqref{2} we get 
\begin{align}
	(2+1)^{k+1}&=\sm{i=0}{k+1}{k+1\choose i}+\sm{i=1}{k+1}{k+1\choose i}2^i, \nonumber\\
	&= 1+\sm{i=0}{k}{k+1\choose i}(2^i+1^i).\label{3}
\end{align}
similarly we get 
\begin{align*}
	4^{k+1}&=(3+1)^{k+1}=\sm{i=1}{k+1}{k+1\choose i}3^i,\\
	&= 3^{k+1}+\sm{i=0}{k}{k+1\choose i}3^i, \\
	&= 1+\sm{i=1}{k}{k+1\choose i}(3^i+2^i+1^i). \text{ (from \eqref{3})}
\end{align*}
continuing in the same way until $ (n+1)^{k+1} $ we get
\begin{align}
	(n&+1)^{k+1}=n^{k+1}+\sm{i=1}{k}{k+1\choose i}n^i,\nonumber\\
	&= 1+\sm{i=1}{k}{k+1\choose i}(n^i+(n-1)^i\ck +2^i+1^i),\nonumber\\
	&=1+\sm{i=1}{k}{k+1\choose i}\left(\sm{a=1}{n}a^i\right).\label{4}
\end{align}
now rewriting equation \eqref{4} to get the term $ \sm{a=1}{n}a^k $ we get the theorem.
\end{proof}
    
\end{multicols}
\end{document}
